\documentclass[a4paper,14pt]{article}
\usepackage[utf8x]{inputenc}
\usepackage[T2A]{fontenc}
\usepackage{float}
\usepackage{cite}
\usepackage[english,russian]{babel}
\usepackage{amssymb,amsmath,amsfonts}
\usepackage{color}
\usepackage{enumerate}
\usepackage[dvips]{graphicx}
\usepackage{setspace}
\usepackage{xcolor}
\usepackage{fancyhdr}

\textheight=220mm
\textwidth=160mm
\oddsidemargin=0.1in
\evensidemargin=0.1in

\begin{document}

\pagestyle{fancy}
\fancyhead{}
\fancyhead[L]{\footnotesize{Курсовая работа 2}}
\fancyhead[R]{\footnotesize{Подготовительная работа}}
\fancyfoot{}
\fancyfoot[L]{\footnotesize{Екатеринбург, Россия}}
\fancyfoot[R]{\footnotesize{}}
\renewcommand{\footrulewidth}{0.1 mm}

\begin{center}
\textbf{Описание полуколец над моноидом разбиений с пересечением}\\
\vspace{\baselineskip}
Глинских Г.А.\\
\emph{Уральский федеральный университет, Екатеринбург, Россия}\\glnskkh@vk.com
\vspace{\baselineskip}\\

\vspace{\baselineskip}
Кузнецов В.М.\\
\emph{Уральский федеральный университет, Екатеринбург, Россия}\\kuznetsovvm88@gmail.com
\vspace{\baselineskip}\\
\end{center}
\vspace{\baselineskip}

% Текст тезиса.

\medskip

\textbf{Вопрос 1 (Волков).} {\sl Отождествим элементы нескольких нижних D-классов моноида разбиений с нулем. Какие элементы составляют полукольцо в этом случае?}

Вспомогательные вопросы:
\begin{enumerate}
    \item Какие элементы при умножении падают на класс ниже?
\end{enumerate}

Идеи:
\begin{enumerate}
    \item Полукольцо над идемпотентами может быть устроено проще.
    \item Можно рассмотреть трехмерные таблицы.
\end{enumerate}

\medskip


\end{document}
